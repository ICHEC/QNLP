\section{Representative Corpora}
\label{sec:representative_corpora}
For the purposes of our implementation, we have opted for some simple corpora examples. Given the ambiguity and complexity of many natural languages, we have decided to maintain a simple data set for use during our development stages.

To follow the DisCo model as proposed, we opt for the commonly discussed sentence structure of \textsc{noun-verb-noun}~\cite{Zeng_Coecke_2016,Coecke_Sadrzadeh_Clark_2010} for the majority of our development and testing. We choose a small sample of public domain and custom phrases to analyse for this type of sentence structure in increasing levels of difficulty:
\begin{itemize}
    \item User-defined simple \textsc{noun-verb-noun} statements (e.g. ``Mary likes fish, John likes cake''). Follows example given in Section~\ref{par:example_hamming}.
    \item Jack and Jill (nursery rhyme). Largely follows simple structure, though more complex structure thant previous example.
    \item Peter Pan (Page 1, paragraph 1). Most complex example. Large departure from \textsc{noun-verb-noun} style.
\end{itemize}
For the early stages of implementation and development the above listed text corpora should suffice to allow us methods to evaluate the quantum-NLP functionality and performance. While we expect longer corpora to be utilised later in the project, we currently think it best to choose those at a later date based on some realistic test cases involving the implemented methods. While a general purpose solution would be ideal, it may be required to adapt the longer corpora based on criteria we later determine from the implementations themselves.



%For the next stage of implementation and working with the models, it 