\section{Mapping DisCo Algorithms on Intel Quantum Simulator}
\label{sec:mapping_disco_algorithms_on_intel_quantum_simulator}
In this section, we discuss the details related to the implementation of the algorithms presented in Section~\ref{sec:abstraction_of_disco_algorithms}. The encoding of classical data into a quantum system can be achieved through a variety of means, though they can be mapped to two different approaches: state (digital) encoding, or amplitude (analogue) encoding~\cite{Schuld_2017,Mitarai_Kitagawa_Fujii_2019}. For our work on the QNLP project the use of a state encoding method for corpus representation offered the best mapping to our problem. Though, much of the problem can be described and implemented using this digital approach, the use of amplitude-based methods are well suited when results are to be obtained. We aim to operate in a mixed-mode approach: encoding and querying our data using state encoding, and obtaining the best result through amplitude modification and measurement.

The nearest-neighbour and sentence similarity algorithms rely on the encoding of a large data-set into the quantum domain, and will follow the approaches discussed in Algorithms~\ref{algo:tagging}, \ref{algo:encoding}, \ref{algo:closest_vector}, and \ref{algo:csc}. Upon these layers, we will implement quantum state engineering and control methods as discussed in the work of Zeng and Coecke~\cite{Zeng_Coecke_2016}, and supplemented by Wiebe et al.~\cite{Wiebe_Kapoor_Svore_2014} and Trugenburger~\cite{Trugenberger_2001,Trugenberger_2002}. At a high level, these methods will allow the discovery of the closest encoded vector (state) to a given test vector (state), and additionally the similarity between meanings of an encoded sentence structure, and that of a test.

%---------------------------------------------------------------------------------
%---------------------------------------------------------------------------------

% \subsubsection{Design considerations}
\subsection{Design Considerations}
\label{sec:design_considerations}
The following discussion applies to the implementation of the methods proposed in Algorithms~\ref{algo:tagging} and \ref{algo:encoding}, as well as extensions to the closest vector and sentence similarity solutions presented in Algorithms~\ref{algo:closest_vector} and \ref{algo:csc}. Figure~\ref{fig:qnlp_controlflow} illustrates the control flow of the planned implementation.

\begin{figure}[!ht]
    \centering
    \includegraphics[width=\textwidth,keepaspectratio]{D1_1/Images/QNLP_Controlflow.pdf}
    \caption{Control flow of proposed QNLP project implementation.}
    \label{fig:qnlp_controlflow}
\end{figure}

The design of this QNLP software solution is based on a two-level approach: $(i)$ the corpus analysis and formatting layer, and $(ii)$ the quantum processing layer. Layer $(i)$ will be written in Python, to avail of ease-of-use in text processing and analysis. Since this layer will be called at the initial outset of the program run, the performance impact will be negligible compared to the DisCo methods. Layer $(ii)$ will be written entirely in C++, and will directly leverage the Intel Quantum Simulator (qHiPSTER). The pre-processed data from layer $(i)$ will be loaded, and encoded into qHiPSTER to generate the full meaning-space of the analysed corpus.

%---------------------------------------------------------------------------------

\paragraph{Pre-computation:}
To analyse and prepare the corpus data for encoding into the quantum state-space, we have determined that the use of well-defined classical routines for corpus tokenisation and tagging can be achieved using the NLTK suite~\cite{BirdKleinLoper09}. Following the pre-compute step, all required data and meta-data for the qHiPSTER encoding and processing will be stored in an intermediary data format. This approach follows the methods discussed by Algorithm~\ref{algo:tagging}.

%---------------------------------------------------------------------------------

\paragraph{Alternative approach to QRAM methods:}\label{par:qram_alt_approach} The methods proposed by Zeng and and Coecke~\cite{Zeng_Coecke_2016} rely on the use of a quantum random access memory (QRAM)~\cite{Giovannetti_Lloyd_Maccone_2008} model to achieve the data access bounds proposed for the closest vector and sentence similarity problems. However, the existence and use of realisable QRAM models in quantum computing algorithms is currently assumed, yet unimplemented in practice. This is in part due to the difficulties in realising the QRAM model as originally proposed (3-level quantum states required), or the large resource requirements necessary for a qubit-implementable model~\cite{Arunachalam_Gheorghiu_Jochym-OConnor_Mosca_Srinivasan_2015}. This renders it almost unusable for real-world problems in current physical or simulated quantum computing systems with low numbers of qubits. Due to the limited number of qubits which we aim to work with, we deemed it necessary to find an alternative approach to realising the state-creation and distance methods, and subsequently Algorithms~\ref{algo:closest_vector} and \ref{algo:csc}. For the purpose of state creation, we aim to follow the approach of Trugenberger~\cite{Trugenberger_2001,Trugenberger_2002}, encoding the data by ``carving-off'' probability slices and encoding the required state information into these slices. Following this, we use an additional qubit register to store a test data vector to compare with. We make use of the Hamming distance metric between the encoded data and the test input to determine the closest encoded state.

This method allows us to avoid the use of a QRAM implementation, at the cost of an iterative state-preparation routine, as we feed in each bit-string to be encoded. The Hamming distance takes advantage of the inherent quantum parallelism offered by the register, and as such can be computed over the entire superposition state simultaneously. 

At this stage, we have several options to determine the closest vector to a known data set: follow the approaches of Trugenberger~\cite{Trugenberger_2001,Trugenberger_2002}, Schuld et al.~\cite{Schuld_Sinayskiy_Petruccione_2014}, Wiebe et al.~\cite{Wiebe_Kapoor_Svore_2014}, or follow alternative approaches defined by use of the the Durr-Hoyer (DH) algorithm~\cite{Durr_Hoyer_1996} or phase shifting. These alternative approach are discussed in further detail in Appendices~\ref{app:hamming_dh} and ~\ref{app:ry_phase}.

It may be instructive to implement many of the above mentioned approaches to determine the optimal method for the closest vector problem, using the encoding strategy and similarity calculations from \cite{Trugenberger_2001,Trugenberger_2002}, the amplitude-based approaches proposed separately by \cite{Schuld_Sinayskiy_Petruccione_2014} and \cite{Wiebe_Kapoor_Svore_2014}, as well as the modified Hamming approaches discussed above. Given any successful implementation of these methods for the closest vector problem, the extension to the sentence similarity problem is trivial. 

We can now discuss the elements of these methods as we see them mapping to our problem.

%---------------------------------------------------------------------------------
%---------------------------------------------------------------------------------

% \subsubsection{Encoding of Classical Bit Stings to Superposition of States}
\subsection{Encoding of Classical Bit Strings to Superposition of States}
\label{sec:encoding_trugenberger}
As discussed above in Section~\ref{sec:design_considerations}, the sequential methods for encoding binary strings to a superposition of states proposed by Trugenberger~\cite{Trugenberger_2001,Trugenberger_2002} will be used for encoding the binary representation of our corpus into a superposition of quantum states.

For a set of $N$ binary patterns $p^i = \{p_1^i, \dots, p_n^i \} $ for $i=1,\dots,N$ of length $n$ being encoded, we can encode these patterns into a quantum superposition of states such that

\begin{align}
    \label{eqn:superposition_state}
    \vert m \rangle = \frac{1}{\sqrt{N}}\sum\limits_{i=1}^ {N} \vert p^i \rangle.
\end{align}

Trugenberger developed a sequential method for creating the superposition $\vert m \rangle$ in \cite{Trugenberger_2001}, by first loading a pattern into an auxiliary register, which was then copied into $\vert m \rangle$, the memory qubit register. This process being repeated for each binary pattern to encode, ``carving'' off the newly encoded state term from the already encoded terms while maintaining the appropriate normalization factors in the amplitudes. This method required $2n + 2$ qubits. It will be referred to as method $1$. 

Trugenberger generalizes this method in \cite{Trugenberger_2002} which generates the superposition of states $\vert m \rangle$, by applying a unitary matrix to the state $\vert 0_1\dots 0_n \rangle$. The process is repeated for each input binary pattern using the aforementioned ``carving'' off method to store the new term in a state. This method requires only $n+2$ qubits, which is a significant reduction from method 1. This will be referred to as method $2$.

The main idea of both methods is detailed below:\\

\noindent
\begin{minipage}[b]{0.45\textwidth}
\textbf{Method 1:}
\begin{align*}{}
\vert \psi_0 \rangle & =  \vert p_1^1,\dots,p_n^1\rangle\vert 01\rangle\vert0_1,\dots,0_n \rangle\\
\vert \psi_1 \rangle & =  \prod\limits_{j=1}^{n} \textrm{2XOR}_{p_j^i u_2 m_j} \vert \psi_0 \rangle\\
\vert \psi_2 \rangle & =  \prod\limits_{j=1}^{n} \textrm{NOT}_{m_j} \textrm{XOR}_{p_j^i m_j}  \vert \psi_1 \rangle\\
\vert \psi_3 \rangle & =  \textrm{nXOR}_{m_1\dots m_n u_1}  \vert \psi_2 \rangle\\
\vert \psi_4 \rangle & =  \textrm{CS}_{u_1 u_2}^{p+1-i}  \vert \psi_3 \rangle\\
\vert \psi_5 \rangle & =  \textrm{nXOR}_{m_1\dots m_n u_1}  \vert \psi_4 \rangle\\
\vert \psi_6 \rangle & =  \prod\limits_{j=n}^{1} \textrm{XOR}_{p_j^i m_j} \textrm{NOT}_{m_j}  \vert \psi_5 \rangle\\
\vert \psi_7 \rangle & =  \prod\limits_{j=n}^{1} 2\textrm{XOR}_{p_j^i u_2 m_j} \vert \psi_6 \rangle
\end{align*}
\hspace{10pt}\\
\hspace{10pt}
\end{minipage}
\begin{minipage}[b]{0.45\textwidth}
\textbf{Method 2:}
Let us define the single qubit unitary gate
\begin{align*}{}
U_j^i = \cos\left(\frac{\pi}{2} p_j^i\right)1 + \sin \left(\frac{\pi}{2} p_j^i\right)\sigma_y.
\end{align*}
For a pattern $p^i$, we can encode the pattern into a single state by applying $U_j^i$;
\begin{align*}
    \vert p^i \rangle = \prod\limits_{j=1}^n U_j^i \vert 0_1\dots 0_n\rangle.
\end{align*}

Then to store it in a superposition state given initial state $\vert\psi_0\rangle$, we compute the following;
\begin{multline*}
    \vert\psi_0 \rangle = \vert 0_1\dots0_n\rangle\vert00\rangle\\
    \vert\psi_1 \rangle = \prod\limits_{j=1}^{n} \left[\left( \textrm{CP}_{u_2}^{i}\right)^{-1} \textrm{NOT}_{u_1} \textrm{CS}_{u_1u_2}^{p+1-i} \textrm{XOR}_{u_{2}u_{1}} \textrm{CP}_{u_{2}}^{i}\right] \\ \times \textrm{NOT}_{u_2} \vert\psi_0\rangle.   
\end{multline*}
\end{minipage}

Repeating either process appropriately for different inputted binary patterns allows us to build an encoded data superposition state in a reliable manner.

It is observed that method $2$ is more efficient than method 1 in terms of the number of qubits required. Method $2$ also requires less operations than method $1$, hence is less computationally expensive. Therefore, method $2$ appears to be the best approach out of the two to use as an encoding strategy. For the sake of completeness and for comparison, both methods will be implemented.

Both of the aforementioned methods are realizable using the Intel Quantum Simulator since they use well defined quantum gate operations including the NOT, XOR, 2XOR (Toffoli) and CU (controlled unitary) gates. We have also developed some necessary extensions of the standard routines available in Intel's Quantum Simulator, including the $n$-qubit controlled unitary gate (nCU), which is required in the form of an n-qubit controlled not (nCX) gate for method 1. Hence, both methods prove to be reliable approaches to proceed with. To implement either method, the following unitary matrix must be first constructed for $i = 1,\dots,N$.
\begin{align*}
    S^i = 
    \begin{bmatrix}
        \sqrt{\frac{i-1}{i}} & \frac{1}{\sqrt{i}} \\
        -\frac{1}{\sqrt{i}}  & \sqrt{\frac{i-1}{i}}
    \end{bmatrix},
\end{align*}
$S^i$ is applied using the controlled unitary operator with $S^i$ being the corresponding unitary matrix. The quantum operations required to implement both methods are displayed above. Each operation used is directly implementable using the Intel Quantum Simulator, apart from nXOR which required an extension of the available gate operations.

%---------------------------------------------------------------------------------
%---------------------------------------------------------------------------------

% \subsubsection{Hamming Distance}\label{sub:hamming_dist}
\subsection{Hamming Distance for Sentence Similarity}\label{sub:hamming_dist}
Two approaches to calculate the Hamming distance are proposed. One, being Trugenberger's, the other a variation of Trugenberger's Hamming distance subroutine developed by Schuld et al.~\cite{Trugenberger_2001, Schuld_Sinayskiy_Petruccione_2014}. For a known test word represented as a binary state, we can calculate the Hamming distance between this word's representation and every other basis word in the superposition. The approach pulls the Hamming distance into the eigenvalue of the eigenstate corresponding to each individual term in the superposition. Thus, the Hamming distance can be stored in the amplitude of its corresponding state. This enables us to quantify the similarity in meaning between the test word and the set of basis words, resulting in the test word's meaning.

The algorithm is detailed in Dirac notation as defined by Trugenberger \cite{Trugenberger_2001} as follows. Given the superposition of encoded binary patterns $\vert m \rangle$ as defined in Eq.~\eqref{eqn:superposition_state} which will be denoted as register $m$, we define two extra registers $d$ of length $n$ and $c$ of length $1$. Thus a total of $2n + 1$ qubits are required for this stage of the calculation. Note, that the ancilla register of the previous encoding strategy can be reused here, reducing the overhead of the number of qubits used. We define our initial state to be the tensor product of these three registers, where $c$ is in the state $\vert 0 \rangle$ and the register $d$ is the binary test pattern $x = x_1,\dots,x_n$. Thus,

\begin{align*}
    \vert \psi_0 \rangle &= \frac{1}{\sqrt{N}}\sum\limits_{i=1}^{N} \vert x \rangle \vert p^i \rangle \vert 0 \rangle \\
    &= \frac{1}{\sqrt{N}}\sum\limits_{i=1}^{N} \vert x_1\dots x_n\rangle\vert p_{1}^{i}\dots p_{n}^{i}\rangle \vert 0 \rangle
\end{align*}

The next step is to apply a Hadamard operation to the ancillary qubit;
\begin{align*}
    \vert \psi_1 \rangle = \frac{1}{\sqrt{2N}}\sum\limits_{i=1}^{N} \vert x_1\dots x_n\rangle\vert p_{1}^{i}\dots p_{n}^{i}\rangle \vert 0 \rangle + \frac{1}{\sqrt{2N}}\sum\limits_{i=1}^{N} \vert x_1\dots x_n\rangle\vert p_{1}^{i}\dots p_{n}^{i}\rangle \vert 1 \rangle
\end{align*}

The Hamming distance between $x$ and $p^i$ for $j=1,\dots, n$ is then calculated in binary format and stored in the $d$ register overwriting the test pattern stored there. This is done by flipping all qubits in register $d$, then applying a controlled-NOT gate on each qubit in register $d$ with the corresponding qubit in $m$ acting as control;

\begin{align*}
    \vert \psi_2 \rangle &= \prod\limits_{j=1}^{n}\textrm{X}_{x_j} \textrm{CNOT}_{x_{j}p_{j}^{i}} \vert \psi_1 \rangle\\
    &= \frac{1}{\sqrt{2N}}\sum\limits_{i=1}^{N} \vert d_1^i\dots d_n^i\rangle\vert p_{1}^{i}\dots p_{n}^{i}\rangle \vert 0 \rangle + \frac{1}{\sqrt{2N}}\sum\limits_{i=1}^{N} \vert d_1^i\dots d_n^i\rangle\vert p_{1}^{i}\dots p_{n}^{i}\rangle \vert 1 \rangle
\end{align*}

\noindent where $d_j^i$ is the distance between the test pattern and the $i^{\textrm{th}}$ training pattern for their $j^{\textrm{th}}$ qubit. Due to a technique used later in the algorithm, it is desired that similar patterns will have a large Hamming distance, hence we say

\begin{align*}
    d_j^i =
    \begin{cases}
        1 \quad \textrm{if } \vert v_j^i \rangle = \vert p_j^i \rangle,\\
        0 \quad \textrm{otherwise}
    \end{cases}
\end{align*}

\noindent which is why the \textrm{NOT} gate was applied to the $m$ register.

The next step is to pull the Hamming distance into the amplitude of its corresponding state. This is done by applying the Hamiltonian

\begin{align*}
    \mathcal{H} = \sum\limits_{j=1}^{n}\left(\frac{\sigma_z + 1}{2}\right)_{d_j} \otimes 1 \otimes \left(\sigma_z\right),
\end{align*}

\noindent through the unitary operator 
\begin{align*}
    U = e^{-\imath \frac{\pi}{2n}\mathcal{H}}.
\end{align*}

Thus we obtain the state

\begin{align*}
    \vert \psi_3 \rangle &= e^{-\imath \frac{\pi}{2n}\mathcal{H}} \vert \psi_2 \rangle \\
    &= \frac{1}{\sqrt{2N}}\sum\limits_{i=1}^{N} e^{\imath \frac{\pi}{2n}d_\mathcal{H}\left(\Vec{x},\vec{p^i}\right)} \vert d_1^i\dots d_n^i\rangle\vert p_{1}^{i}\dots p_{n}^{i}\rangle \vert 0 \rangle + \frac{1}{\sqrt{2N}}\sum\limits_{i=1}^{N} e^{-\imath \frac{\pi}{2n}d_\mathcal{H}\left(\Vec{x},\Vec{p^i}\right)} \vert d_1^i\dots d_n^i\rangle\vert p_{1}^{i}\dots p_{n}^{i}\rangle \vert 1 \rangle,
\end{align*}

\noindent where $d_\mathcal{H}\left(\Vec{x},\Vec{p^i}\right)$ is the Hamming distance between $\Vec{x}$ and $\Vec{p^i}$ with the closer the value is to $n$, the more similar the patterns are to each other.

Finally, applying a Hadamard to the control register $c$, the new state reduces to

\begin{align*}
    \vert \psi_4 \rangle = \frac{1}{\sqrt{2N}}\sum\limits_{i=1}^{N} \cos\left[ \frac{\pi}{n}d_\mathcal{H}\left(\Vec{x},\vec{p^i}\right) \right] \vert d_1^i\dots d_n^i\rangle\vert p_{1}^{i}\dots p_{n}^{i}\rangle \vert 0 \rangle \\ + \frac{1}{\sqrt{2N}}\sum\limits_{i=1}^{N} \sin\left[ \frac{\pi}{n}d_\mathcal{H}\left(\Vec{x},\Vec{p^i}\right)\right] \vert d_1^i\dots d_n^i\rangle\vert p_{1}^{i}\dots p_{n}^{i}\rangle \vert 1 \rangle.
\end{align*}

Due to the nature of measurement of a quantum system it is not possible to obtain the amplitude of each state directly. Instead, we can obtain the probability of each quantum state (training pattern) occurring $\mathcal{P}\left( p^i\right)$. This probability is influenced by the Hamming distance in such a way that higher probabilities correspond to a higher Hamming distance, indicating that the test pattern is close to that particular training pattern.

To conduct this measurement, we must first collapse the ancillary cubit in register $c$. If it collapses to $\vert0\rangle$ with a high probability, then the test pattern is very different from all of the training patterns. Otherwise, if it collapses to $\vert 0 \rangle$ with a high probability, then the test pattern is close to the training patterns. More formally,

\begin{align*}
    \mathcal{P}\left(\vert c \rangle = \vert 0 \rangle \right) &= \frac{1}{N} \sum\limits_{i = 1}^{N} \cos^2 \left[\frac{\pi}{2n} d_\mathcal{H}\left(\Vec{x},\vec{p^i}\right)\right], \\
    \mathcal{P}\left(\vert c \rangle = \vert 1 \rangle \right) &= \frac{1}{N} \sum\limits_{i = 1}^{N} \sin^2 \left[\frac{\pi}{2n} d_\mathcal{H}\left(\Vec{x},\vec{p^i}\right)\right].
\end{align*}

The probability of each individual pattern occurring can be obtained by repeating the experiment for a large number of times, measuring the register $m$ corresponding to the training pattern, and recording the observed state of this register. Thus a distribution of probabilities of each state occurring can be obtained. Furthermore, we know that 

\begin{align*}
    \mathcal{P}\left(\vert m \rangle = \vert p^i \rangle \right) = \frac{1}{N\mathcal{P}\left(\vert c \rangle = \vert 0 \rangle \right)} \cos^2 \left[\frac{\pi}{2n} d_\mathcal{H}\left(\Vec{x},\vec{p^i}\right)\right],    
\end{align*}

\noindent and so the Hamming distance can be obtained. Note that the slightly different method used by Schuld et al. \cite{Schuld_Sinayskiy_Petruccione_2014} associates each training pattern with a class which is also encoded into the superposition. Measurement is then conducted on the class qubits rather than the training pattern qubits. These methods will also be considered as trivial extensions to Trugenberger's method.

It is also worth noting that Trugenberger proposed a version of the Hamming distance memory retrieval algorithm which uses only a unitary operator to fuse the standard quantum gate operations in a similar way as discussed in method $1$ of Section \ref{sec:encoding_trugenberger}. This method, or a slight modification of it, will also be examined to determine on which performs best.  

%---------------------------------------------------------------------------------
%---------------------------------------------------------------------------------

\iftrue
%\iffalse
\paragraph{Hamming distance similarity example:}\label{par:example_hamming}
The method introduced here demonstrates the use of the Hamming distance as discussed earlier in Section~\ref{sub:hamming_dist}, to outline a sample closest vector (state) problem.
We define a set of orthogonal words (i.e. words with no overlapping meaning) through which we can define the meaning of other words. For this test case, we choose a simple vocabulary. Following the work of Coecke et al., we opt for the simplified \textsc{noun-verb-noun} sentence structure, and define sets of words within each of these spaces, through which we can construct our full meaning space. For nouns, we have two sets, $n_s = \{\textsc{adult},\textsc{child},\textsc{smith},\textsc{surgeon}\}$ and $n_o = \{\textsc{outside},\textsc{inside}\}$, and for verbs, $v = \{\textsc{stand},\textsc{sit},\textsc{move},\textsc{sleep}\}$. With this \textsc{noun-verb-noun} meaning space sentence structure, we can calculate this as

\begin{equation*}
\left(
\begin{array}{c}
\textsc{adult} \\\\
\textsc{child} \\\\
\textsc{smith} \\\\
\textsc{surgeon} \\\\
\end{array}
\right) \otimes
\left(
\begin{array}{c}
\textsc{stand} \\\\
\textsc{sit} \\\\
\textsc{move} \\\\
\textsc{sleep} \\\\
\end{array}
\right) \otimes
\left(
\begin{array}{c}
\textsc{outside} \\\\
\textsc{inside} \\\\
\end{array}
\right)
\end{equation*}

For the above meaning-space, we can begin to define entities that exist in this space, and choose to encode them into quantum states as follows:
\begin{itemize}
    \item $\textsc{John is an adult, and a smith}$. The state is defined as \\ $\vert \textsc{John} \rangle = 1/\sqrt{2}\left(\vert \textsc{adult} \rangle + \vert \textsc{smith} \rangle\right)$, which is a superposition of the number of matched entities from the basis set.
    \item $\textsc{Mary is a child, and a surgeon}$. The state is defined as \\ $\vert \textsc{Mary} \rangle = 1/\sqrt{2}\left(\vert \textsc{child} \rangle + \vert \textsc{surgeon} \rangle\right)$, again, following the same procedure as above.
\end{itemize}

\noindent With these definitions, we can look at some sentences with \textsc{John} and \textsc{Mary}. An example corpus with meaning that can exist in our meaning-space is:
\textsc{John rests outside, and Mary walks inside}.

In this instance, we also require meanings for \textsc{rests} and \textsc{walks}. If we examine synonyms for \textsc{rests} and cross-compare with our chosen vocabulary, we can find \textsc{sit} and \textsc{sleep}. Similarly, for \textsc{walks} we can have \textsc{stand} and \textsc{move}. We can define the states of these words as $\vert \textsc{rest} \rangle = 1/\sqrt{2}\left(\vert \textsc{sit} \rangle + \vert \textsc{sleep} \rangle\right)$ and $\vert \textsc{walk} \rangle = 1/\sqrt{2}\left(\vert \textsc{stand} \rangle + \vert \textsc{move} \rangle\right)$.
Now that we have a means to define the states in terms of our vocabulary, we can begin constructing quantum states to encode the data: firstly, however, we must decide on how to encode the states.

We can give each entity in the sets of nouns and verbs a unique binary index, where the word can be mapped to the binary string (gives the encoding value), and inversely mapped back (decoding value). Classically, we can use a key-value pair for this task, wherein the key (word) gives the value (index), and vice versa. As we intend to encode this data into quantum states, we opt for a sufficient number of qubits to define our states: namely, we require $\lceil \log_2({\textrm{num elements})} \rceil$ qubits to represent our data from each respective set.

For this example, we opt for the following mappings:

\begin{equation*}
\begin{array}{c|c|c}
\mathbf{Dataset} & \mathbf{Word} & \mathbf{Bin.~Index} \\
\hline
{n_s} & \textsc{adult} & 00 \\
{n_s} & \textsc{child} & 01 \\
{n_s} & \textsc{smith} & 10 \\
{n_s} & \textsc{surgeon} & 11 \\
\\
{v} & \textsc{stand} & 00 \\
{v} & \textsc{move} & 01 \\
{v} & \textsc{sit} & 10 \\
{v} & \textsc{sleep} & 11 \\
\\
{n_o} & \textsc{inside} & 0 \\
{n_o} & \textsc{outside} & 1 \\
\end{array}
\end{equation*}

It should be noted that here we have chosen a na\"ive mapping for our key-value pairs; the use of a Gray code~\cite{gray_code}, or alternative encoding strategies, may provide more instructive meanings to the meaning-space queries later. However, we leave this to future work for implementation and testing, and use the proposed encoding method for this example.
To encode the meaning of the statement \textsc{John rests outside, and Mary walks inside}, we define the following quantum states:

\begin{equation*}
\begin{array}{c|c|c}
\mathbf{Dataset} & \mathbf{Word} & \mathbf{State} \\
\hline
{n_s} & \textsc{John} & (\vert 00 \rangle + \vert 10 \rangle)/\sqrt{2} \\
{n_s} & \textsc{Mary} & (\vert 01 \rangle + \vert 11 \rangle)/\sqrt{2} \\
{v} & \textsc{walk} & (\vert 00 \rangle + \vert 01 \rangle)/\sqrt{2} \\
{v} & \textsc{rest} & (\vert 10 \rangle + \vert 11 \rangle)/\sqrt{2} \\
{n_o} & \textsc{inside} & \vert 0 \rangle  \\
{n_o} & \textsc{outside} & \vert 1 \rangle  \\
\end{array}
\end{equation*}

We construct our quantum state by tensoring these entities, following the \textsc{noun-verb-noun} DisCo-like formalism. If we consider the \textsc{John} and \textsc{Mary} sentences separately for the moment, they are respectively given by the states $(1/2)*(\vert 00 \rangle + \vert 10 \rangle)\otimes (\vert 10 \rangle + \vert 11 \rangle)\otimes \vert 0 \rangle$ for John, and $(1/2)*(\vert 01 \rangle + \vert 11 \rangle)\otimes (\vert 00 \rangle + \vert 01 \rangle)\otimes \vert 1 \rangle$ for Mary. Tidying these up and multiplying out gives 

\begin{equation*}
\begin{array}{ccc}
\textsc{John rests outside} & \rightarrow & \frac{1}{2}(\vert 00100 \rangle + \vert 00110 \rangle +\vert 10100 \rangle +\vert 10110 \rangle) \rightarrow \vert J\rangle, \\
\\
\textsc{Mary walks inside} & \rightarrow & \frac{1}{2}(\vert 01001 \rangle + \vert 01011 \rangle +\vert 11001 \rangle +\vert 11011 \rangle) \rightarrow \vert M\rangle,\\
\end{array}
\end{equation*}
where the full meaning is given by $\vert S\rangle = \frac{\vert J \rangle + \vert M \rangle}{\sqrt{2}}$, which is a superposition of the 8 unique encodings defined by our meaning-space and sentence. 

From here, we can now introduce an additional register of equal size to $\vert S\rangle$, wherein we will encode a test vector for comparison against the encoded meanings, denoted as $\vert T \rangle$. This test vector will be overwritten with the Hamming distance of it against the other states, with the resulting value giving the intended meaning, or closest possible meaning to that of the test. Continuing with our example, we use ``Adults stand inside'', which is encoded as $\vert T \rangle = \vert 00000 \rangle$. Constructing our query state, $\vert Q \rangle = \vert T \rangle \otimes \vert S\rangle$, we get the following fully expanded state

\begin{eqnarray*}
\vert Q \rangle = \frac{1}{2/\sqrt{2}} (\vert 00000 \rangle\vert 00100 \rangle + \vert 00000 \rangle\vert 00110 \rangle + \vert 00000 \rangle\vert 10100 \rangle + \vert 00000 \rangle\vert 10110 \rangle \\+\vert 00000 \rangle\vert 01001 \rangle + \vert 00000 \rangle\vert 01011 \rangle + \vert 00000 \rangle\vert 11001 \rangle + \vert 00000 \rangle\vert 11011 \rangle).
\end{eqnarray*}
Marking the Hamming distance between both registers, and overwriting the test register modifies the above state to
\begin{eqnarray*}
\vert Q^{\prime} \rangle = \frac{1}{2/\sqrt{2}} (\vert 00100 \rangle\vert 00100 \rangle + \vert 00110 \rangle\vert 00110 \rangle + \vert 10100 \rangle\vert 10100 \rangle +  \vert 10110 \rangle\vert 10110 \rangle \\+\vert 01001 \rangle\vert 01001 \rangle + \vert 01011 \rangle\vert 01011 \rangle + \vert 11001 \rangle\vert 11001 \rangle + \vert 11011 \rangle\vert 11011 \rangle).
\end{eqnarray*}

This example is simple as any state differing from $\vert 00\cdots\rangle$ flips the test when a $\vert 1\rangle$ is seen in the training. The data in register $\vert T \rangle$ now encodes the number of different flips between both test and training data; thus, it can be seen as a similarity measure, and the state with the fewest flips will give the closest meaning to the posed test. The above example has only a single data set with one flip, that of $\vert 00100 \rangle$. By decoding the result using the mapping provided earlier, we can see that this corresponds with a meaning of \textsc{Adult(s) sit inside}. We can subsequently see that this data will give a non-zero overlap with $\vert J \rangle$.

The above approach can be mapped directly onto a qubit register, with indices used for partitioning into sub-registers. As such, the above methods can be directly implemented on qHiPSTER.

%There are some remaining questions on implementing this, however. Firstly, ensuring that the Hamming distance is encoded correctly, and such that the amplitudes are adjusted to respect the minimal difference has the highest probability of outcome. Additionally, if we wish to compare that statement with the previous corpus, we need to determine which statement encoded the resulting data-set --- a problem unless we aim to calculate many overlap integrals.

\fi
%---------------------------------------------------------------------------------
%---------------------------------------------------------------------------------
