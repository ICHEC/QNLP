\section{Dependency graph of QNLP solution}\label{app:dep_graph}
The structure of the QNLP solution is given by the depdendency graph in Figure~\ref{fig:qnlp_deps}. 
Here the vertices coloured red are C++ classes, blue are python packages/modules, and green are external dependencies with the gradient colour indicating whether they are for the Python or C++ modules.

\begin{figure}[h!]
    \centering
    \includegraphics[width=\textwidth]{Images/qnlp_deps.pdf}
    \caption{Red $\rightarrow$ C++; Blue $\rightarrow$  Python; Green $\rightarrow$ Ext. library.}
    \label{fig:qnlp_deps}
\end{figure}

Rather than the solution discussed previously (in deliverable D1.1) of using an intermediary SQLite3 database to communicate the results between the C++ layer compute layer and the Python pre-processing layer, by connecting both layers through the Python interpreter allows a single ended application to be created. This is demonstrated with the Jupyter notebook presented in Appendix~\ref{qnlp-example}, wherein all work was carried out through the Python interpreter, and through use pybind11, calls were made to the relevant methods in the C++ layer.
\clearpage